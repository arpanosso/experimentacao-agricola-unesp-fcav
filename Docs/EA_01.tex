% Options for packages loaded elsewhere
\PassOptionsToPackage{unicode}{hyperref}
\PassOptionsToPackage{hyphens}{url}
%
\documentclass[
]{article}
\usepackage{lmodern}
\usepackage{amssymb,amsmath}
\usepackage{ifxetex,ifluatex}
\ifnum 0\ifxetex 1\fi\ifluatex 1\fi=0 % if pdftex
  \usepackage[T1]{fontenc}
  \usepackage[utf8]{inputenc}
  \usepackage{textcomp} % provide euro and other symbols
\else % if luatex or xetex
  \usepackage{unicode-math}
  \defaultfontfeatures{Scale=MatchLowercase}
  \defaultfontfeatures[\rmfamily]{Ligatures=TeX,Scale=1}
\fi
% Use upquote if available, for straight quotes in verbatim environments
\IfFileExists{upquote.sty}{\usepackage{upquote}}{}
\IfFileExists{microtype.sty}{% use microtype if available
  \usepackage[]{microtype}
  \UseMicrotypeSet[protrusion]{basicmath} % disable protrusion for tt fonts
}{}
\makeatletter
\@ifundefined{KOMAClassName}{% if non-KOMA class
  \IfFileExists{parskip.sty}{%
    \usepackage{parskip}
  }{% else
    \setlength{\parindent}{0pt}
    \setlength{\parskip}{6pt plus 2pt minus 1pt}}
}{% if KOMA class
  \KOMAoptions{parskip=half}}
\makeatother
\usepackage{xcolor}
\IfFileExists{xurl.sty}{\usepackage{xurl}}{} % add URL line breaks if available
\IfFileExists{bookmark.sty}{\usepackage{bookmark}}{\usepackage{hyperref}}
\hypersetup{
  pdftitle={Experimentação Agrícola - UNESP/FCAV},
  pdfauthor={Panosso A.R.},
  hidelinks,
  pdfcreator={LaTeX via pandoc}}
\urlstyle{same} % disable monospaced font for URLs
\usepackage[margin=1in]{geometry}
\usepackage{color}
\usepackage{fancyvrb}
\newcommand{\VerbBar}{|}
\newcommand{\VERB}{\Verb[commandchars=\\\{\}]}
\DefineVerbatimEnvironment{Highlighting}{Verbatim}{commandchars=\\\{\}}
% Add ',fontsize=\small' for more characters per line
\usepackage{framed}
\definecolor{shadecolor}{RGB}{248,248,248}
\newenvironment{Shaded}{\begin{snugshade}}{\end{snugshade}}
\newcommand{\AlertTok}[1]{\textcolor[rgb]{0.94,0.16,0.16}{#1}}
\newcommand{\AnnotationTok}[1]{\textcolor[rgb]{0.56,0.35,0.01}{\textbf{\textit{#1}}}}
\newcommand{\AttributeTok}[1]{\textcolor[rgb]{0.77,0.63,0.00}{#1}}
\newcommand{\BaseNTok}[1]{\textcolor[rgb]{0.00,0.00,0.81}{#1}}
\newcommand{\BuiltInTok}[1]{#1}
\newcommand{\CharTok}[1]{\textcolor[rgb]{0.31,0.60,0.02}{#1}}
\newcommand{\CommentTok}[1]{\textcolor[rgb]{0.56,0.35,0.01}{\textit{#1}}}
\newcommand{\CommentVarTok}[1]{\textcolor[rgb]{0.56,0.35,0.01}{\textbf{\textit{#1}}}}
\newcommand{\ConstantTok}[1]{\textcolor[rgb]{0.00,0.00,0.00}{#1}}
\newcommand{\ControlFlowTok}[1]{\textcolor[rgb]{0.13,0.29,0.53}{\textbf{#1}}}
\newcommand{\DataTypeTok}[1]{\textcolor[rgb]{0.13,0.29,0.53}{#1}}
\newcommand{\DecValTok}[1]{\textcolor[rgb]{0.00,0.00,0.81}{#1}}
\newcommand{\DocumentationTok}[1]{\textcolor[rgb]{0.56,0.35,0.01}{\textbf{\textit{#1}}}}
\newcommand{\ErrorTok}[1]{\textcolor[rgb]{0.64,0.00,0.00}{\textbf{#1}}}
\newcommand{\ExtensionTok}[1]{#1}
\newcommand{\FloatTok}[1]{\textcolor[rgb]{0.00,0.00,0.81}{#1}}
\newcommand{\FunctionTok}[1]{\textcolor[rgb]{0.00,0.00,0.00}{#1}}
\newcommand{\ImportTok}[1]{#1}
\newcommand{\InformationTok}[1]{\textcolor[rgb]{0.56,0.35,0.01}{\textbf{\textit{#1}}}}
\newcommand{\KeywordTok}[1]{\textcolor[rgb]{0.13,0.29,0.53}{\textbf{#1}}}
\newcommand{\NormalTok}[1]{#1}
\newcommand{\OperatorTok}[1]{\textcolor[rgb]{0.81,0.36,0.00}{\textbf{#1}}}
\newcommand{\OtherTok}[1]{\textcolor[rgb]{0.56,0.35,0.01}{#1}}
\newcommand{\PreprocessorTok}[1]{\textcolor[rgb]{0.56,0.35,0.01}{\textit{#1}}}
\newcommand{\RegionMarkerTok}[1]{#1}
\newcommand{\SpecialCharTok}[1]{\textcolor[rgb]{0.00,0.00,0.00}{#1}}
\newcommand{\SpecialStringTok}[1]{\textcolor[rgb]{0.31,0.60,0.02}{#1}}
\newcommand{\StringTok}[1]{\textcolor[rgb]{0.31,0.60,0.02}{#1}}
\newcommand{\VariableTok}[1]{\textcolor[rgb]{0.00,0.00,0.00}{#1}}
\newcommand{\VerbatimStringTok}[1]{\textcolor[rgb]{0.31,0.60,0.02}{#1}}
\newcommand{\WarningTok}[1]{\textcolor[rgb]{0.56,0.35,0.01}{\textbf{\textit{#1}}}}
\usepackage{longtable,booktabs}
% Correct order of tables after \paragraph or \subparagraph
\usepackage{etoolbox}
\makeatletter
\patchcmd\longtable{\par}{\if@noskipsec\mbox{}\fi\par}{}{}
\makeatother
% Allow footnotes in longtable head/foot
\IfFileExists{footnotehyper.sty}{\usepackage{footnotehyper}}{\usepackage{footnote}}
\makesavenoteenv{longtable}
\usepackage{graphicx,grffile}
\makeatletter
\def\maxwidth{\ifdim\Gin@nat@width>\linewidth\linewidth\else\Gin@nat@width\fi}
\def\maxheight{\ifdim\Gin@nat@height>\textheight\textheight\else\Gin@nat@height\fi}
\makeatother
% Scale images if necessary, so that they will not overflow the page
% margins by default, and it is still possible to overwrite the defaults
% using explicit options in \includegraphics[width, height, ...]{}
\setkeys{Gin}{width=\maxwidth,height=\maxheight,keepaspectratio}
% Set default figure placement to htbp
\makeatletter
\def\fps@figure{htbp}
\makeatother
\setlength{\emergencystretch}{3em} % prevent overfull lines
\providecommand{\tightlist}{%
  \setlength{\itemsep}{0pt}\setlength{\parskip}{0pt}}
\setcounter{secnumdepth}{-\maxdimen} % remove section numbering

\title{Experimentação Agrícola - UNESP/FCAV}
\author{Panosso A.R.}
\date{17/08/2020}

\begin{document}
\maketitle

\hypertarget{introduuxe7uxe3o-ao-curso}{%
\section{1 - INTRODUÇÃO AO CURSO}\label{introduuxe7uxe3o-ao-curso}}

A Estatística Experimental tem como objetivo o estudo dos experimentos,
ist é, seu planejamneto, execução, análise dos dados e interpretação dos
resultados obtidos.

Para que um experimentador conduza e avalie uma pesquisa corretamente, é
essencial um certo conhecimento de estatística, principalmente no que se
refere às potencialidades e às limitações das técnicas utilizadas. Assim
sendo, o curso de Experimentação Agrícola visa apresentar aos alunos os
métodos estatística mais usado em Agronomia.

Vejamos então, alguns conceitos básicos necessários para um bom
entendimento da estatística Experimental

\hypertarget{alguns-conceitos-buxe1sico}{%
\subsection{1.1. ALGUNS CONCEITOS
BÁSICO}\label{alguns-conceitos-buxe1sico}}

\hypertarget{populauxe7uxe3o}{%
\subsubsection{1.1.1 POPULAÇÃO}\label{populauxe7uxe3o}}

Boa parte do conhecimento humano está baseado em um número relativamente
reduzido de informações. Isto é verdadeiro, tanto no que se refere aos
problemas do cotidiano, como no que se refere à pesquisa científica.

Por definição \textbf{POPULAÇÃO} é o conjunto de elementos que têm em
comum uma determinada característica. Todo o subconjunto não vazio e com
menor número de elemento do que o conjunto definido como
\textbf{população} constitui, por definição, uma \textbf{AMOSTRA} desta
população.

Uma população em ecologia é o número total de indivíduos de uma
determinada espécie em uma área definida. Por exemplo, o número total de
lagartas de \emph{Spodoptera frugiperda} em uma cultura de milho
constitui uma população. Esta população, embora finita, é considerada
para fins de amostragem como uma população infinita.

Uma vez definida a unidade amostral (1 planta, um conjunto de 5 plantas,
ou um quadrado no qual será contado o número de lagartas), a população
pode ser considerada como um conjunto de unidade amostrais e um
subconjunto tomado aleatoriamente deste subconjunto é chamado de
\textbf{AMOSTRA ALEATÓRIA DE TAMANHO N}.

Assim sendo, as observações são obtidas através de contagens do número
de indivíduos em cada unidade amostral. Está observações são chamadas de
\textbf{VARIÁVEL EM ESTUDO}.

\hypertarget{tratamento}{%
\subsubsection{1.1.2 TRATAMENTO}\label{tratamento}}

É o método, elemento ou material, cujo efeito se deseja medir ou
comparar em um experimento. Por exemplo, um tratamento pode ser: Uma
variedade de cana-de-açúcar, um híbrido de sorgo, uma dose de um adubo
para a cultura do milho, um espaçamento par a cultura do algodão, um
recipiente para produção de mudas de eucalipto, um inseticida para
controle de pragas, etc.

\hypertarget{experimento-ou-ensaio}{%
\subsubsection{1.1.3 EXPERIMENTO OU
ENSAIO}\label{experimento-ou-ensaio}}

Experimento é um trabalho previamente planejado, no qual se faz
comparação dos efeitos dos tratamentos.

\hypertarget{unidade-experimental-ou-parcela}{%
\subsubsection{1.1.4 UNIDADE EXPERIMENTAL OU
PARCELA}\label{unidade-experimental-ou-parcela}}

É a unidade na qual o tratamento é aplicado. É na parcela que obtemos os
dados que deverão refletir o efeito de cada tratamento no ensaiado. A
pArcela pode ser constituída por uma planta, uma área om um grupo de
plantas, uma placa de petri com um meio de cultura, um animal, um lote
de animais, etc.

\hypertarget{delineamento-experimental}{%
\subsubsection{1.1.5 DELINEAMENTO
EXPERIMENTAL}\label{delineamento-experimental}}

É o plano utilizado na experimentação, e implica na forma como os
tratamento deverão ser distribuídos nas unidades experimentais e como
serão analisados os dados a serem obtidos. Como exemplo, temos o
delineamento inteiramente casualizado (DIC), o delineamento em blocos
casualizados (DBC) o delineamento em quadrado latino (DQL), entre
outros.

\hypertarget{medidas-de-posiuxe7uxe3o-e-de-dispersuxe3o}{%
\subsection{1.2 MEDIDAS DE POSIÇÃO E DE
DISPERSÃO}\label{medidas-de-posiuxe7uxe3o-e-de-dispersuxe3o}}

As populações são descritas por certas características chamadas de
\textbf{Parâmetros}. A amostras são descritas pelas mesmas
características, que, neste caso, são chamadas de \textbf{Estimativas
ded Parâmetros}, \textbf{Estatísticas da amostra}. Alguns destes
larâmetros são chamado de \textbf{medidas de posição} e outros de
\textbf{medida de dispersão}

\hypertarget{medidas-de-posiuxe7uxe3o-ou-de-tenduxeancia-central}{%
\subsubsection{1.2.1. MEDIDAS DE POSIÇÃO OU DE TENDÊNCIA
CENTRAL}\label{medidas-de-posiuxe7uxe3o-ou-de-tenduxeancia-central}}

Uma característica comum a todas as populações ou amostras é a
varaibilidade dos indivíduos que as constituem,

Geralmente, os dados de uma população ou amostra tendem a ser mais
numerosos em torno de um valor central e vão se tornando mais raros à
medida que no afatamos desse valor. A medida de posição representa o
valo em torno do qual os dados observados tendem a se agrupar.

Das medidas de posição, as mais utilizadas são a Média Aritmética, que
pode se definida como:
\texttt{A\ Soma\ de\ todas\ as\ observações\ divididas\ pelo\ número\ delas}

Assim, para uma população com \(N\) elementos, ditos \(x_1\), \(x_2\),
\ldots{} , \(x_N\), a média aritmética será:

\[
m = \frac{\sum_{i=1}^{N}x_i}{N} = \frac{x_1+x_2+...+x_N}{N}
\] Para uma amostra com \(n\) elementos, ditos \(x_1\), \(x_2\),
\ldots{} , \(x_n\), a média aritmética será:

\[
\bar{x}=\hat{m} = \frac{\sum_{i=1}^{N}x_i}{N} = \frac{x_1+x_2+...+x_N}{n}
\] \textbf{exemplo}

Considere como exemplo os dados abaixo, referente à ltura de plantas
daninha (em cm) em uma amostra de 5 plantas em uma área de pastagem.

\begin{longtable}[]{@{}lllll@{}}
\toprule
\(x_1\) & \(x_2\) & \(x_3\) & \(x_4\) & \(x_5\)\tabularnewline
\midrule
\endhead
5 & 3 & 2 & 4 & 3\tabularnewline
\bottomrule
\end{longtable}

Então, para estes dados, a estimativa da média aritmética será:

\[
\hat{m} = \frac{\sum_{i=1}^{N}x_i}{N} = \frac{5+3+2+4+3}{5} = \frac{17}{5} = 3,4\;cm
\] \includegraphics{R.png}

\begin{Shaded}
\begin{Highlighting}[]
\CommentTok{# Defina o vetor de dados}
\NormalTok{X <-}\StringTok{ }\KeywordTok{c}\NormalTok{(}\DecValTok{5}\NormalTok{,}\DecValTok{3}\NormalTok{,}\DecValTok{2}\NormalTok{,}\DecValTok{4}\NormalTok{,}\DecValTok{3}\NormalTok{)}

\CommentTok{# Calculando a soma dos elementos de X (G)}
\NormalTok{G=}\KeywordTok{sum}\NormalTok{(X)}
\NormalTok{G}
\end{Highlighting}
\end{Shaded}

\begin{verbatim}
## [1] 17
\end{verbatim}

\begin{Shaded}
\begin{Highlighting}[]
\CommentTok{# Encontrando o número total de elementos de X}
\NormalTok{n=}\KeywordTok{length}\NormalTok{(X)}
\NormalTok{n}
\end{Highlighting}
\end{Shaded}

\begin{verbatim}
## [1] 5
\end{verbatim}

\begin{Shaded}
\begin{Highlighting}[]
\CommentTok{# Calculando diretamente a média aritmética}
\KeywordTok{mean}\NormalTok{(X)}
\end{Highlighting}
\end{Shaded}

\begin{verbatim}
## [1] 3.4
\end{verbatim}

A diferença entre um valor observado (\(x_i\)) e a média aritmética
(\(\hat{m}\)) é denominado de desvio (\(d_i\)), ou seja:

\[
 d_i = x_i - \hat{m}, \text{ para  i =1, 2, ...,n  }
\]

\begin{longtable}[]{@{}lllll@{}}
\toprule
\(d_1 = 5-3.4\) & \(d_2=3-3.4\) & \(d_3 = 2-3.4\) & \(d_4 = 4-3.4\) &
\(d_5 = 3-3.4\)\tabularnewline
\midrule
\endhead
1,6 & -0,4 & -1,4 & 0,6 & -0,4\tabularnewline
\bottomrule
\end{longtable}

Por mos mostrar que a soma dos desvios é igual a zero, para qualquer
conjunto de dados.

\[
\sum_{i=1}^{N}d_i = 0
\]

\includegraphics{R.png}

\begin{Shaded}
\begin{Highlighting}[]
\CommentTok{# Calculando os desvios}
\NormalTok{d<-X}\OperatorTok{-}\KeywordTok{mean}\NormalTok{(X)}

\CommentTok{# Apresentando os valores de desvios}
\NormalTok{d}
\end{Highlighting}
\end{Shaded}

\begin{verbatim}
## [1]  1.6 -0.4 -1.4  0.6 -0.4
\end{verbatim}

\begin{Shaded}
\begin{Highlighting}[]
\CommentTok{# Prova de que o A Soma dos Desvios é igual a Zero}
\KeywordTok{round}\NormalTok{(}\KeywordTok{sum}\NormalTok{(d))  }\CommentTok{# a função rond arredonda a saida da função soma para 0 casas decimais}
\end{Highlighting}
\end{Shaded}

\begin{verbatim}
## [1] 0
\end{verbatim}

\hypertarget{medidas-de-dispersuxe3o}{%
\subsubsection{1.2.2. MEDIDAS DE
DISPERSÃO}\label{medidas-de-dispersuxe3o}}

As medidas de dispersão são também chamadas de \textbf{medidas de
variação}, e modem o grau com que os dados tendem a se afastar de uma
valor central, que geralmente é a média aritmética.

Como em todas as amostras (ou populações) ocorre variabilidade dos
elementos que as constituem, amostra com mesma média pode apresentar
distribuições diferentes, e portanto, somente a média não fornece
informação clara de como os dados se distribuem. Assim, para representar
melhor a maneira pela qual os dados se distribuem, são utilizadas as
medidas de dispersão ou de variação.

Dentre as medidas de dispersão, discutiremos a Variância, o Desvio
PAdrão, o Erro padrão da Média e o Coeficiente de variação.

\hypertarget{variuxe2ncia}{%
\subsubsection{1.2.1. VARIÂNCIA}\label{variuxe2ncia}}

A variância é uma medida de dispersão que leva em conta todas as
observações. É indiscutivelmente, a melhor medida de dispersão.

A variância de uma população é representada por \(\sigma^2\), lê-se
sigma dois, e pode ser definida como ``A média dos quadrados dos desvios
de todos os dados em relação à média aritmética''.

Então, para uma população com \(N\) temos:

\[
\sigma^2 = \frac{SQD}{N} = \frac{d_1+d_2+...+d_N}{N} =\frac{\sum_{i=1}^Nd_i^2}{N} \\
\sigma^2 =\frac{\sum_{i=1}^N(x_i-m)^2}{N}
\]

Note que utilizando esta fórmula o cálculo da variancia seria bastante
trabalhoso, no caso de N ser um número muito grande.

Existe porém um método mais prático de se calcular a variância, que pode
ser obtido desenvolvendo-se as fórmula da soma de quadrados dos desvios
(SQD) . Assim, temos:

\[
SQD = \sum_{i=1}^N(x_i-m)^2 = \sum_{i=1}^Nx_i^2-\frac{(\sum_{i=1}^N x_i)^2}{N}
\]

Então, a fórmula simplificada da variância será:

\[
\sigma^2 =\frac{\sum_{i=1}^Nx_i^2-\frac{(\sum_{i=1}^N x_i)^2}{N}}{N}
\]

A vantagem desta fórmula é que trabalhamos diretamente com os dados
originais, não havendo necessidade de clacularmos a média e os desvios
em relação a ela.

O termo \(\frac{(\sum_{i=1}^Nx_i)^2}{N}\) é denomidado de Correção
devido à média ou simplesmente \textbf{Correção}, representado por \(C\)
e é de grande utilização nas futuras análise de variâncias.

Normalmente, na prática, trabalhamos com amostras, e a estimativa da
vairância, representada por \(s^2\), é calculada, para uma
\textbf{amostra} com \(n\) elementos, representados por \(x_1\),
\(x_2\), \ldots,\(x_n\), por:

\[
s^2 = \frac{SQD}{n-1} = \frac{d_1+d_2+...+d_n}{n-1} =\frac{\sum_{i=1}^nd_i^2}{n-1} \\
s^2 =\frac{\sum_{i=1}^n(x_i-\hat{m})^2}{n-1}
\]

ou ainda,

\[
s^2 =\frac{\sum_{i=1}^nx_i^2-\frac{(\sum_{i=1}^n x_i)^2}{n}}{n-1}
\]

\textbf{Observações:}

1 - A variância tem sempre valor positivo, e sua unidade é quadrática.

2 - O denominador utilizado do nálculo da variância é chamado de
\textbf{grau de liberdade} da estimativa da variância, sempre dado por
\(n-1\)

\textbf{exemplo} No exemplo anterior de altura de plantas daninhas:

No caso temos

\begin{longtable}[]{@{}lllll@{}}
\toprule
\(d_1 = 5-3.4\) & \(d_2=3-3.4\) & \(d_3 = 2-3.4\) & \(d_4 = 4-3.4\) &
\(d_5 = 3-3.4\)\tabularnewline
\midrule
\endhead
1,6 & -0,4 & -1,4 & 0,6 & -0,4\tabularnewline
\bottomrule
\end{longtable}

\[
s^2 =\frac{(1,6^2+(-0.4)^2+(-1,4)^2+0,6^2+(-0,4)^2)^2}{5-1} = \frac{5.2}{4} = 1,3 \;cm^2
\]

Pela fórmula que não utiliza os desvios teríamos:

\begin{longtable}[]{@{}lllll@{}}
\toprule
\(x_1\) & \(x_2\) & \(x_3\) & \(x_4\) & \(x_5\)\tabularnewline
\midrule
\endhead
5 & 3 & 2 & 4 & 3\tabularnewline
\bottomrule
\end{longtable}

\[
s^2 =\frac{\sum_{i=1}^nx_i^2-\frac{(\sum_{i=1}^n x_i)^2}{n}}{n-1} = \frac{(5^2+3^2+2^2+4^2+3^2)-\frac{(5+3+2+4+3)^2}{5}}{5-1} = \frac{63-\frac{(17)^2}{5}}{5-1} = \frac{63-57,8}{4}=1,3 \;cm^2
\]

\includegraphics{R.png}

\begin{Shaded}
\begin{Highlighting}[]
\CommentTok{# Forma mais simples de calcular a variância amostral}
\KeywordTok{var}\NormalTok{(X)}
\end{Highlighting}
\end{Shaded}

\begin{verbatim}
## [1] 1.3
\end{verbatim}

\hypertarget{desvio-padruxe3o}{%
\subsubsection{1.2.2. DESVIO PADRÃO}\label{desvio-padruxe3o}}

A variância, pela sua natureza, tem a unidade quadrática. A sua raiz
quadrada, que ainda é uam medida de dispersão é denominada desvio
padrão:

A vantagem do desvio padrão é ter a mesma unidade dos dados originais e,
consequentemente, da média.

É a mais utilizada das medidas de dispersaõ, e é representada por
\(\sigma\) para a população, com estimativa \(s\) para a amostra. Então:

\[
\sigma = \sqrt{\sigma^2} \text{   e    } s = \sqrt{s^2}
\]

\textbf{exemplo} \[
s = \sqrt{1,3} = 1,140175\;cm
\] \includegraphics{R.png}

\begin{Shaded}
\begin{Highlighting}[]
\CommentTok{# Forma mais simples de calcular o desvio padrão}
\KeywordTok{sd}\NormalTok{(X)}
\end{Highlighting}
\end{Shaded}

\begin{verbatim}
## [1] 1.140175
\end{verbatim}

\hypertarget{erro-padruxe3o-da-muxe9dia}{%
\subsubsection{1.2.3. ERRO PADRÃO DA
MÉDIA}\label{erro-padruxe3o-da-muxe9dia}}

Se em vez de uma amostra tivéssemos várias, provenientes de uma mesma
população, obteríamos diversas estimativas da média, e provavelmente
distintas umas das outras.

A partir dessas diversas estimativas da média, poderíamos estimar uma
variância, considerandos-e os desvios de cada média individual, em
relação à média de todas elas. Seria então uma estimativa da variância
das média.

Entretanto, demonstra-se que a partir de uma única amostra, podemos
estimar essa variância, através da fórmula:

\[
Var(\bar{X}) = \hat{V}(\hat{m}) = \frac{s^2}{n}
\] onde \(s^2\) é a estimativa da variância dos \(n\) dados, calculada
de maneira usual

A sua raiz quadrada é denominada Erro Padrão da Média, ou seja:

\[
s(\hat{m}) = \frac{s}{\sqrt{n}}
\]

O erro padrão da média fornece uma idéia da precisão da estimativa da
média, isto é quanto menor ele for, maior precisão terá a estimativa da
média.

Assim para os dados de altura de plantas daninhas temos:

\textbf{exemplo}

\[
s(\hat{m}) = \frac{s}{\sqrt{n}} = \frac{1,140175}{\sqrt{5}} = 0,599\;cm
\]

\includegraphics{R.png}

\begin{Shaded}
\begin{Highlighting}[]
\CommentTok{# Erro padrão da média}
\KeywordTok{sd}\NormalTok{(X)}\OperatorTok{/}\KeywordTok{sqrt}\NormalTok{(n)}
\end{Highlighting}
\end{Shaded}

\begin{verbatim}
## [1] 0.509902
\end{verbatim}

Sempre que apresentarmos uma média, é conveniente apresentar também o
seu erro padrão. Assim, no exemplo poderíamos apresentar a média e o seu
erro padrão, da seguinte maneira:

\[
3,4 \pm 0,599\;cm
\]

Quanto menor o valor do erro padrão da média, mais precisa foi a
estimativa da média.

\hypertarget{coeficiente-de-variauxe7uxe3o}{%
\subsubsection{1.2.4. COEFICIENTE DE
VARIAÇÃO}\label{coeficiente-de-variauxe7uxe3o}}

É uma medida de dispersão que expressa percentualmente o desvio padrão
por unidade de média, ou seja:

\[
CV = \frac{100 \cdot s}{\hat{m}}
\]

Como \(s\) e \(\hat{m}\) são expressos na mesma unidade dos dados, o
coeficiente de variação é um número abstrato, isto é, não tem unidade e
portanto é expresso em porcentagem da média.

No exemplo de altura de plantas daninhas, temos:

\textbf{exemplo}

\[
CV = \frac{100 \cdot s}{\hat{m}} = \frac{100 \cdot 1,140175}{3,4} = 25,80\%
\] \includegraphics{R.png}

\begin{Shaded}
\begin{Highlighting}[]
\CommentTok{# Coeficiente de Variação}
\DecValTok{100}\OperatorTok{/}\KeywordTok{sd}\NormalTok{(X)}\OperatorTok{/}\KeywordTok{mean}\NormalTok{(X)}
\end{Highlighting}
\end{Shaded}

\begin{verbatim}
## [1] 25.79582
\end{verbatim}

Nos ensaios agrícolas de campo, esperam-se coeficientes de variação da
ordem de 10 a 20\%. Porém em ensaior de levantamento de pragas,
normalmente os coeficientes de variação são maiores que 30\%.

\end{document}
